\documentclass[a4paper]{article}

\usepackage[utf8]{inputenc}
\usepackage[margin=1in]{geometry}
\usepackage[bookmarks]{hyperref}
\usepackage{graphicx}
\usepackage{listings}
\usepackage{color}
\usepackage{courier}
\usepackage{enumitem}
\usepackage{pdfpages}
\usepackage{fancyvrb}
\usepackage{amsmath}
\usepackage[toc,page]{appendix}

\definecolor{col_light_grey}{rgb}{0.9, 0.9, 0.9}
\definecolor{col_green}{rgb}{0,0.6,0}
\definecolor{col_grey}{rgb}{0.5,0.5,0.5}
\definecolor{col_mauve}{rgb}{0.58,0,0.82}

% \lstset{
%   escapeinside={\%*}{*}
% }
\lstdefinestyle{default}{
  backgroundcolor=\color{col_light_grey},
  basicstyle=\footnotesize,
  breaklines=true,
  captionpos=b,
  commentstyle=\color{col_green},
  keywordstyle=\color{blue},
  stringstyle=\color{col_mauve},
  tabsize=2
}
\lstset{language=bash}

\hypersetup{
  pdfinfo={
    Title={COMP1204: Database Theory and Practice Coursework},
    Author={Huw Jones},
  },
  colorlinks=false,
  pdfborder=0 0 0,
}

\setlist[itemize]{noitemsep, nosep}
\setlist[enumerate]{noitemsep, nosep}

\pagestyle{headings}

\author{Huw Jones \\27618153}
\title{COMP1204: Database Theory and Practice Coursework}

\begin{document}
\maketitle

\section{ERD and Normalisation}
\subsection{EX1 - Relation}
\begin{Verbatim}[commandchars=+\[\]]
HotelReview(
  +underline[ReviewID:+textbf[Integer]],
  Author:+textbf[String],                 Date:+textbf[Date],
  HotelID:+textbf[Integer],               URL:+textbf[String],
  AveragePrice:+textbf[Integer],          Content:+textbf[String],
  Overall:+textbf[Integer],               OverallRating:+textbf[Integer],
  BusinessService:+textbf[Integer],       CheckIn:+textbf[Integer],
  Cleanliness:+textbf[Integer],           Rooms:+textbf[Integer],
  Service:+textbf[Integer],               Value:+textbf[Integer],
  NoReaders:+textbf[Integer],             NoHelpful:+textbf[Integer]
)
\end{Verbatim}

\subsection{EX2 - Functional Dependencies}
\begin{tabular}{l l l c l l l l}
Author & Date & HotelName & $\to$ & Content & OverallRating & BusinessService & CheckIn \\
&&&& Cleanliness & Rooms & Service & Value \\
&&&& NoReaders & NoHelpful & & \\
HotelID & & & $\to$ & URL & Overall & AveragePrice & \\
\end{tabular}

\subsection{EX3 - Normalised Relations}
\begin{Verbatim}[commandchars=+\[\]]
Hotel(
  +underline[HotelID:+textbf[Integer]],
  URL:+textbf[String],
  OverallRating:+textbf[Integer],
  AveragePrice:+textbf[Integer]
)
Review(
  +underline[ReviewID:+textbf[Integer]],
  Author:+textbf[String],
  Date:+textbf[Date],
  HotelID:+textbf[Integer],
  Content:+textbf[String],
  Overall:+textbf[Integer],
  BusinessService:+textbf[Integer],
  CheckIn:+textbf[Integer],
  Cleanliness:+textbf[Integer],
  Rooms:+textbf[Integer],
  Service:+textbf[Integer],
  Value:+textbf[Integer],
  NoReaders:+textbf[Integer],
  NoHelpful:+textbf[Integer]
)
\end{Verbatim}

\subsection{EX4 - ERD Model}
\begin{center}
\def\svgwidth{\columnwidth}
\input{ERD_Diagram.pdf_tex}
\end{center}

\section{Relation Algebra}
\subsection{EX5 - Finding a user's reviews}
$\sigma_{Author=X}(Review)$
\subsection{EX6 - Finding users with more than two reviews}
$\sigma_{}(Review)$
\subsection{EX7 - Finding all hotels with more than 10 reviews}
\subsection{EX8 - Finding all hotels with overall rating and cleanliness}

\section{SQL}
\subsection{EX9 - Creating HotelReviews Table}
\begin{lstlisting}[language=SQL, style=default]
CREATE TABLE HotelReviews (
  reviewID INTEGER PRIMARY KEY,
  author VARCHAR(256) NOT NULL,
  reviewDate DATE NOT NULL,
  hotelID INTEGER NOT NULL,
  URL VARCHAR(256),
  averagePrice INTEGER,
  content TEXT,
  overall INTEGER NOT NULL,
  overallRating INTEGER NOT NULL,
  businessService INTEGER,
  checkIn INTEGER,
  cleanliness INTEGER,
  rooms INTEGER,
  service INTEGER,
  value INTEGER,
  noReaders INTEGER NOT NULL DEFAULT 0,
  noHelpful INTEGER NOT NULL DEFAULT 0
);
\end{lstlisting}
Please note, I have deliberately avoided using \texttt{AUTOINCREMENT} on the ``reviewID'' column.
This is because the SQLite documentation specifically recommends not using this keyword as it ``should be avoided if not strictly needed''.
\subsection{EX10 - Creating a SQL insert script}
Please see Appendix \ref{appendix:unix_code} for the Unix code.
I chose to implement my script mainly in awk.
As awk supports record/field processing, it was just the case of getting it to correctly identify the records and fields.
\subsection{EX11 - Creating Normalised Tables}
\subsection{EX12 - Populating Normalised Tables}
\subsection{EX13 - Creating Indexes}

\section{Data Retrieval and Analysis}
\subsection{EX14 - Relational Algebra to SQL}

\section{Conclusions}

\begin{appendices}
\section{Unix Code}
\label{unix_code}
\subsection{generatesql.sh}
\lstinputlisting[style=default, language=bash]{generatesql.sh}
\subsection{generatesql.awk}
\lstinputlisting[style=default, language=awk]{generatesql.awk}
\end{appendices}

\end{document}