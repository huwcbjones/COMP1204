\documentclass[12pt,a4paper]{article}

\usepackage[utf8]{inputenc}
\usepackage[bookmarks]{hyperref}
\usepackage{graphicx}
\usepackage{listings}
\usepackage{color}

\definecolor{col_green}{rgb}{0,0.6,0}
\definecolor{col_grey}{rgb}{0.5,0.5,0.5}
\definecolor{col_mauve}{rgb}{0.58,0,0.82}

\lstset{
  backgroundcolor=\color{white},
  basicstyle=\footnotesize,
  breaklines=true,
  captionpos=b,
  commentstyle=\color{col_green},
  escapeinside={\%*}{*)},
  keywordstyle=\color{blue},
  stringstyle=\color{col_mauve},
  tabsize=2,
}
\lstset{language=bash}

\hypersetup{
  pdfinfo={
    Title={COMP1204: Unix Coursework},
    Author={Huw Jones},
  },
  colorlinks=false,
  pdfborder=0 0 0,
}

\pagestyle{headings}

\author{Huw Jones \\27618153}
\title{COMP1204: Unix Coursework}

\begin{document}
\maketitle
\newpage

\section{Scripts}
\subsection{Count Reviews: Single File}
\subsubsection{Source}
\lstinputlisting{Q1/countreviews.sh}
\subsubsection{Explanation}
\begin{lstlisting}
grep -c "<Author>" $1
\end{lstlisting}
As every review has an author, it made sense to search for the ``\textless Author\textgreater" string.
Using the ``-c" argument for grep returns a search count, rather than the string of occurrences.
``\$1" is used to get the first (real) argument when the script is executed on the command line.

\subsection{Count Reviews: Directory}
\lstinputlisting{Q2/countreviews.sh}

\subsection{Count Reviews: Sorted}
\lstinputlisting{Q3/countreviews.sh}

\section{Hypothesis Testing}

\section{Discussion}

\end{document}