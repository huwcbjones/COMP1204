\documentclass[a4paper]{article}

\usepackage[utf8]{inputenc}
\usepackage[margin=1in]{geometry}
\usepackage[bookmarks]{hyperref}
\usepackage{graphicx}
\usepackage{listings}
\usepackage{color}

\definecolor{col_light_grey}{rgb}{0.9, 0.9, 0.9}
\definecolor{col_green}{rgb}{0,0.6,0}
\definecolor{col_grey}{rgb}{0.5,0.5,0.5}
\definecolor{col_mauve}{rgb}{0.58,0,0.82}

\lstset{
  backgroundcolor=\color{col_light_grey},
  basicstyle=\footnotesize,
  breaklines=true,
  captionpos=b,
  commentstyle=\color{col_green},
  escapeinside={\%*}{*)},
  keywordstyle=\color{blue},
  stringstyle=\color{col_mauve},
  tabsize=2,
}
\lstset{language=bash}

\hypersetup{
  pdfinfo={
    Title={COMP1204: Unix Coursework},
    Author={Huw Jones},
  },
  colorlinks=false,
  pdfborder=0 0 0,
}

\pagestyle{headings}

\author{Huw Jones \\27618153}
\title{COMP1204: Unix Coursework}

\begin{document}
\maketitle
\newpage

\section{Scripts}

%	3.1.1
%	Question 1
%
\subsection{Count Reviews: Single File}

\begin{lstlisting}
grep -c "<Author>" $1
\end{lstlisting}
As every review has an author, it made sense to search for the ``\textless Author\textgreater" string.
Using the ``-c" argument for grep returns a search count, rather than the string of occurrences.
``\$1" is used to get the first (real) argument when the script is executed on the command line.

%	3.1.1
%	Question 2
%
\subsection{Count Reviews: Directory}

\begin{lstlisting}
function getReviewCount () {
	grep -c "<Author>" $1
}
\end{lstlisting}
This function returns the number of reviews in a hotel data file.

\begin{lstlisting}
if [ -d $1 ]
then

...
fi
\end{lstlisting}
This checks whether the argument passed was a directory, and if so, execute the appropriate code.

\begin{lstlisting}
for file in $1/*
do
	getReviewCount $file
done;
\end{lstlisting}
This section is executed if the argument passed as a directory.
It iterates over the files in the directory, then calls the ``getReviewCount" function for each file.

\begin{lstlisting}
else
	getReviewCount $1
fi
\end{lstlisting}
This code is executed if the argument passed to the script is not a directory.
It executes the ``getReviewCount" as it should normally do for a file. \newline \newline
This script works for both 3.1.1: 1 and 2.

%	3.1.1
%	Question 3
%
\subsection{Count Reviews: Sorted}
\begin{lstlisting}
reviewCount=""
\end{lstlisting}
Initialises variable ``reviewCount" to an empty string.

\begin{lstlisting}
function getTrimmedHotelFile() {
	echo $1 | sed -e 's:^.*\/::' -e 's:.dat::'
}
\end{lstlisting}
This functions uses sed substitution to remove all directory prefixes up to the last forward slash in the file path (/) from ``\$file".
It then uses another substitution to remove the .dat extension.
Then, it assigns the result to ``hotelName".

The regex works as follows.
\begin{lstlisting}
^
\end{lstlisting}
Matches from the start of the string
\begin{lstlisting}
^.*
\end{lstlisting}
Matches 0 or more all characters from the start of the string.
\begin{lstlisting}
\/
\end{lstlisting}
Matches a forward slash. The backslash is used to escape it as forward slash is a special character.

\begin{lstlisting}
^[/a-zA-Z_0-9\-]*\/
\end{lstlisting}
So, overall this regex matches all directories up to the last slash before the file name.
This allows the script to remove the directory prefix.

\begin{lstlisting}
currentCount="$hotelName"$'\t'"$(getReviewCount $file)"
\end{lstlisting}
Sets ``currentCount" to the hotel ID and the output of ``getReviewCount".
The tab is used to make the output prettier than just using spaces.

\begin{lstlisting}
reviewCount="$reviewCount"$'\n'"$currentCount"
\end{lstlisting}
Appends ``currentCount" to the ``reviewCount" using a newline so we can maintain 1 hotel per line.

\begin{lstlisting}
echo -e "$reviewCount" | sort -k2nr
\end{lstlisting}
Echoes the contents of ``reviewCount" to sdtOut whilst maintaining escape characters.
The contents of ``reviewCount" is then piped into the stdIn of sort, which reverse sorts the hotel list by number of reviews.
Sort takes a ``-k" argument that is used to specify the sort column.
Here, ``-k2nr" sorts the string by the second column (number of reviews)(``k2"), as a numeric type (``n"), reversely (``r") (to sort by the greater number first).

%	3.1.2
%	Hypothesis Testing Script
%
\newpage
\subsection{Average Reviews}
\begin{lstlisting}
# Gets the average review of the hotel
function getAverageScore() {
	echo -e "$(getScores $1)" | awk '
	BEGIN {
		TotalScore=0;
		n=0;
	}

	{
		TotalScore += $0;
		n++;
	}

	END {
		printf ("%.2f\n", (TotalScore / n));
	}
	'
}
\end{lstlisting}
This function uses ``awk" to calculate the mean.
First, it gets a list of scores and pipes it to awk.
Then, it instantiates ``TotalScore" and ``n" to 0.
Next, it loops through the lines (of scores), adds the score to the total, then increments ``n".
Finally, it uses ``printf" to print the result of the $\frac{\sum{x}}{n}$ ($\bar{x}$) to 2 decimal places.


\begin{lstlisting}
# Checks argument passed was a directory
if [ -d $1 ]
then
	# Loops through all files and prints HOTEL_ID AVERAGE_REVIEW
	for file in $1/*
	do
		currentHotel="$(getTrimmedHotelFile $file)"$'\t'"$(getAverageScore $file)"
		hotels="$hotels"$'\n'"$currentHotel"
	done
	
	# Sort hotels by second column (rating)
	echo -e "$hotels" | sort -k2nr
else
	echo "$(getTrimmedHotelFile $1) $(getAverageScore $1)"
fi
\end{lstlisting}
This part of the script checks a directory was passed as the first argument.
Otherwise, it loops over every file in the data directory, then prints the hotel ID with the average review score, sorted by average review.

%	3.1.2
%	Hypothesis Testing Script 2
%
\newpage
\subsection{Statistical Significance}

\begin{lstlisting}
function getN() {
	echo -e "$(getScores $1)" | wc -l
}
\end{lstlisting}
Returns the number of reviews a hotel has.

\begin{lstlisting}
function getSD() {
	file="$1"
	mean="$2"
	echo -e "$(getScores $1)" | awk '
		BEGIN {
			Total=0;
			n=0;
		}

		{
			Total += (($0 - '$mean')^2);
			n++;
		}

		END {
			var = Total/(n-1);
			printf("%.9f\n", sqrt(var));
		}
	'
}
\end{lstlisting}
This function calculates the standard deviation (for a sample).
It uses the formula $\sigma = \frac{\sum{(x_i - \bar{x})^2}}{(n-1)}$.
However, to maintain accuracy, it does not round the result to 2 decimal places, this is done later on when the result is printed to the terminal.

\begin{lstlisting}
# Calculates the common standard deviation
function getSx1x2() {
	S2X1=$1
	nX1=$2
	S2X2=$3
	nX2=$4

	awk '
	BEGIN {
		numer=( ('$nX1' - 1) * ('$S2X1')^2) + ( ('$nX2' - 1) * ('$S2X2')^2);
		denom=( '$nX1' + '$nX2' - 2 );
		frac = numer / denom;
		printf ("%.9f\n", sqrt(frac) );
	}
	'
}
\end{lstlisting}
This function calulates $s X_1 X_2$ and takes 4 arguments: $S^2X_1$, $n_1$, $S^2X_2$, and  $n_2$.
It uses awk to calulate $s X_1 X_2 = \sqrt{\frac{(n_1 - 1)s^{2}_{X1} + (n_2 - 1)s^{2}_{X2}}{n_1 + n_2 - 2}}$.

\begin{lstlisting}
# Calculates the t-statistic
function getT_Statistic() {
	M1=$1
	n1=$2
	M2=$3
	n2=$4
	Sx1x2=$5
	awk '
	BEGIN {
		numer=( '$M1' - '$M2' );
		denom=( '$Sx1x2' * sqrt((1/'$n1') + (1/'$n2')) );
		printf ("%.9f\n", (numer/denom) );
	}
	'
}
\end{lstlisting}
Using the formula $t = \frac{\bar{X_1} - \bar{X_2}}{s X_1 X_2 \cdot \sqrt{\frac{1}{n_1} + \frac{1}{n_2}}}$, this function calculates the t-statistic.
It takes 4 arguments, ``M1"/``M2" ($\bar{X_1}$, $\bar{X_2}$) and ``n1"/``n2" ($n_1$, $n_2$).

\begin{lstlisting}
function round(){
	awk 'BEGIN { printf ("%.'$2'f\n", '$1') }'
}
\end{lstlisting}
Rounds a float to a given number of decimal places.
It takes 2 arguments, ``\$1", the number to round, and ``\$2", the number of places.

\newpage
\section{Hypothesis Testing}
\subsection{Hypotheses}
\begin{equation}
H_{0} : \textrm{The hotels are not statistically significant}
\end{equation}
\begin{equation}
H_{A} : \textrm{The hotels are statistically significant}
\end{equation}

\subsection{Using}
\begin{itemize}
	\item Two tailed test, $\textrm{critical value} = 1.965$
	\item 5\% significance
\end{itemize}

\subsection{Outcome}
\begin{lstlisting}[language={}]
$ ./statistical_sig.sh hotel_188937 hotel_203921
t: 0.03
Mean hotel_188937:      4.78,   SD: 0.63
Mean hotel_203921:      4.78,   SD: 0.53
0
\end{lstlisting}
Therefore null hypothesis validated.
Two hotels are not statistically significant.

\newpage
\section{Discussion}

\subsection{Storage}
TripAdvisor currently uses a tag based format to save their reviews.
The format requires the use of one file per hotel.

\subsubsection{Format}
Comments are provided using double forward slash (C style).
\begin{lstlisting}[language={}]
// Hotel Identifier
<Overall Rating>n
<Avg. Price>$nnn
<URL>xxx
					// newline
<Author>
<Content>
<Date>MMM d, YYYY
<img>				//optional
<No. Reader>n
<No. Helpful>n
<Overall>
<Value>
<Rooms>
<Location>
<Cleanliness>
<Check in / front desk>
<Service>
<Business service>
\end{lstlisting}
This format is very open for interpretation.
For example, there is no end deliminator for each field.
Since there seems to be no concrete specification of how the data is to be interpreted, software could assume that the end of a line signifies the end of a field.
However, there are fields that span multiple lines - namely the ``Content" field.

Compared to the JSON or XML specifications, there is no logical structure to the data.
Structuring your data properly reduces the time to process it, and makes it easier to comprehend; not only for humans, but for machines too.

\subsubsection{Repetition}
Secondly, storing the data in this format creates a lot of repetition.
For each review, the tag names are specified.
With hotels in the hundreds of thousands and each hotel having a multiple reviews, there is a lot of duplication.
On a scale this large, that amount of data duplication should be avoided.

A Relational Database Management System (RDBMS) would be beneficial in storing this data.
Depending on the schema selected, this would impose some limits (e.g.: maximum content length), however they should not affect the storage of reviews compared to the current system.

\subsubsection{Data Relationships}
An RDBMS would allow the individual hotel details to be migrated away from being stored with reviews.
Not only would this allow for the schema to integrate with the rest of the site, it would allow the reviews to be separated.

A foreign key constraint (FKC) could be used to maintain a relationship with the review and the hotel.
Additionally, the review could also be linked to a user table.
Linking reviews to specific users could also introduce the possibility of a reputation-based system.

\subsubsection{Reputation}
A reputation based system could offer the ability for reviewers with more, higher quality reviews to affect the bias of a hotel's ranking (similar to StackOverflow, but better).
User reputation could also be used to filter trustworthy, or untrustworthy reviews.
This means bots, or spamming users, would be able to be filtered out from the pool of trustworthy reviews.
On the other hand, this would impeded new users' ability to raise their reputation due to the competition with existing reviewers (a problem StackOverflow encounters).

The sytem should also take into account the number of reviews, average rating, and user reputation in order to produce a metric that can rank hotels accurately and as unbiased as possible (within reason).
Take Amazon for example, it could recommend a product with 2, 5 star reviews.
However, is a product with 2, 5 star reviews better than a product with 2,000 reviews and an average rating of 4.5?
Somehow the system would need to balance this to allow well reviewed hotels to not take precedance over slightly less known, but equally as well reviewed hotels.
Due to the complex nature of this system, caching hotel rankings and periodically updating them, would offer better performance then calculating a hotel's rank everytime it's page is requested.

\subsubsection{Speed}
RDBMS's can also be orders of magnitude faster than the existing file based system.
This means querying or performing statistical analysis on the data (as completed in this report) would be able to be completed with a greater speed and ease.
RDBMS provides Structured Query Language (SQL) as a means of querying the database.
SQL also contains functions, and can be used to create functions that perform the statistical anaylsis.


\subsection{Collection and Authenticity}
Collecting reviews at first would seem a rather simple affair, however it requires more thought than just putting a review form on a webpage.
Many factors need to be considered when moderating reviews.
Currently, the form shows no sign of bot protection.

\subsubsection{Authenticity}
Connecting with third party holiday/booking sites could be used the validate the authenticity of a review.
As the user has booked a holiday through the third party, the system can verify that a user has visited the hotel, and therefore can make sure that users have actually visited the hotel they are reviewing.
This could be connected to the hotel rating system where verified reviews (where the user can prove they have been there) would hold more weight than reviews that are unverified.

Adding a visit date field to the review form could also help users get a better representation of a hotel.
For example, a bad hotel may be bought and run under new management.
In this example, you would expect reviews to increase, by logging the date of a stay, the hotel review could be adjusted to take into account the date when this change in reviews occured.
This would mean older, lower, or higher rated reviews would have less of an affect compared to newer reviews that fitted this new pattern.

\subsubsection{Bot and Spam Prevention}
So how can one validate the authenticity of a review?
Checking for bots could be offloaded to the reCAPTCHA service.
This would instantly filter out bots using Google's reCAPTCHA service.
reCAPTCHA uses machine learning to identify organic (human) responses to small puzzles or visual identification challenges, and filter out bots.

Restricting the rate at which users can post reviews could aid trustworthyness.
This restriction would hinder users that are paid to create false reviews as it would restrict the rate at which they could post reviews.

\subsubsection{Moderation}
Machine learning algorithms could also be used to moderate a users content.
With a record of a user's past reviews, and current hotel reviews, a statistical model could be produced that can indicate whether a review (based on the individual rating components) is withing the tolerances for that hotel.
If the review is flagged by the system because it does not fit the existing pattern, it could be raised to the attention of moderators.

Moderators are staff members whose job is to moderate reviews.
Reviews could be flagged for moderation through a report/flag review button, by the review rating not fitting the statistical pattern (see above), or as a way to validate new users.

To prevent lots of new accounts being created to get around the anti-bot system, flagging the first `x' reviews of a new user to be moderated could help to reduce the amount of spam and low quality reviews.
This system would hold the review to be pended for moderation.
A moderator would then review the review and either accept it, reject it, or if it's a low quality review, provide some feedback on how to increase the quality of the review.

\end{document}