\documentclass[a4paper]{article}

\usepackage[utf8]{inputenc}
\usepackage[margin=1in]{geometry}
\usepackage[bookmarks]{hyperref}
\usepackage{graphicx}
\usepackage{listings}
\usepackage{color}

\definecolor{col_light_grey}{rgb}{0.9, 0.9, 0.9}
\definecolor{col_green}{rgb}{0,0.6,0}
\definecolor{col_grey}{rgb}{0.5,0.5,0.5}
\definecolor{col_mauve}{rgb}{0.58,0,0.82}

\lstset{
  backgroundcolor=\color{col_light_grey},
  basicstyle=\footnotesize,
  breaklines=true,
  captionpos=b,
  commentstyle=\color{col_green},
  escapeinside={\%*}{*)},
  keywordstyle=\color{blue},
  stringstyle=\color{col_mauve},
  tabsize=2,
}
\lstset{language=bash}

\hypersetup{
  pdfinfo={
    Title={COMP1204: Unix Coursework},
    Author={Huw Jones},
  },
  colorlinks=false,
  pdfborder=0 0 0,
}

\pagestyle{headings}

\author{Huw Jones \\27618153}
\title{COMP1204: Unix Coursework}

\begin{document}
\maketitle
\newpage

\section{Scripts}

%	3.1.1
%	Question 1
%
\subsection{Count Reviews: Single File}

\subsubsection{Source}
\lstinputlisting{Q1/countreviews.sh}


\subsubsection{Explanation}

\begin{lstlisting}
grep -c "<Author>" $1
\end{lstlisting}
As every review has an author, it made sense to search for the ``\textless Author\textgreater" string.
Using the ``-c" argument for grep returns a search count, rather than the string of occurrences.
``\$1" is used to get the first (real) argument when the script is executed on the command line.

%	3.1.1
%	Question 2
%
\subsection{Count Reviews: Directory}

\subsubsection{Source}
\lstinputlisting{Q2/countreviews.sh}


\subsubsection{Explanation}

\begin{lstlisting}
function reviewcount () {
	grep -c "<Author>" $1
}
\end{lstlisting}
I turned the count reviews into a function to use later.

\begin{lstlisting}
if [ -d $1 ]
then

...
fi
\end{lstlisting}
This checks whether the argument passed was a directory, and if so, execute the appropriate code.

\begin{lstlisting}
for file in $1/*
do
	reviewcount $file
done;
\end{lstlisting}
This section is executed if the argument passed as a directory.
It iterates over the files in the directory, then calls the ``reviewcount" function for each file.

\begin{lstlisting}
else
	reviewcount $1
fi
\end{lstlisting}
This code is executed if the argument passed to the script is not a directory.
It executes the ``reviewcount" as it should normally do for a file. \newline \newline
This script works for both 3.1.1: 1 and 2.

%	3.1.1
%	Question 3
%
\newpage
\subsection{Count Reviews: Sorted}
\lstinputlisting{Q3/countreviews.sh}

\subsubsection{Explanation}
\begin{lstlisting}
reviewCount=""
\end{lstlisting}
Initialises variable ``reviewCount" to an empty string.

\begin{lstlisting}
hotelName=${file#$1/}
\end{lstlisting}
This removes the directory prefix from ``\$1" (from when the script was called) from ``\$file" and assigns it to ``hotelName".

\begin{lstlisting}
hotelName=${hotelName%.dat}
\end{lstlisting}
Removes the ``.dat"" suffix from ``hotelName" and assigns it to ``hotelName".

\begin{lstlisting}
currentCount="$(reviewcount $file) $hotelName"
\end{lstlisting}
Sets ``currentCount" to the output of ``reviewCount" with the hotel name appended to it. This makes it easier to use sort.

\begin{lstlisting}
reviewCount="$reviewCount"$'\n'"$currentCount"
\end{lstlisting}
Appends ``currentCount" to the ``reviewCount" using a newline so we can maintain 1 hotel per line.

\begin{lstlisting}
echo -e "$reviewCount" | sort -nr | awk '{print $2 " " $1}'
\end{lstlisting}
Echoes the contents of ``reviewCount" to sdtOut whilst maintaining escape characters.
The contents of ``reviewCount" is then piped into the stdIn of sort, which reverse sorts the hotel list by number of reviews.
Finally, awk prints out the sorted list in the format [hotel] [review count].
The list was originally stored as [review count] [hotel] as this makes it easier to use sort as we don't have to specify a sort column.
Using awk to print out the list as required was child's play.

%	3.1.2
%	Hypothesis Testing Script
%
\newpage
\subsubsection{Average Reviews}
\lstinputlisting{averagereviews.sh}

\newpage
\section{Hypothesis Testing}

\newpage
\section{Discussion}


\end{document}